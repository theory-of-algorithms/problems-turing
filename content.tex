%!TEX root = problems.tex


\begin{questions}

\question
Consider the following Turing Machine.
\begin{table}[H]
  \centering
  \begin{tabular}{ccccc}
	State & Read & Write & Move & Next \\
	0 & B & B &  & Fail \\
	0 & 0 & 0 & L & 0 \\
	0 & 1 & 1 & L & 1 \\

	0 & B & B &  & Accept \\
	0 & 0 & 0 & L & 1 \\
	0 & 1 & 1 & L & 0 \\
  \end{tabular}
\end{table}
Determine what happens when the Turing Machine is run with the following inputs initially on the tape.
\begin{parts}
  \part 0001
  \part 0111
  \part 0110
  \part 0101010001
  \part 00000000000000111
  \part 00
  \part 1
\end{parts}


\begin{solution}
\begin{parts}
  \part 11
\end{parts}
\end{solution}


\question
Give the state table for a Turing Machine that appends a parity bit to a tape with a string of consecutive 0's and 1's}


\begin{solution}
\end{solution}

\question
Construct a Turing Machine to compute the sequence $0\_1\_0\_1\_0\_1\ldots$, that is, 0 blank 1 blank 0 blank, etc.

\end{questions}

\begin{thebibliography}{12}
  
\bibitem{simplyscheme}
  Brian Harvey and Matt Wright,
  \emph{Simply Scheme: Introducing Computer Science},
  MIT,
  1999.
  
\bibitem{projecteuler}
  Project Euler,
  \emph{Project Euler},
  2016.

\end{thebibliography}
