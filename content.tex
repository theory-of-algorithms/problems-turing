%!TEX root = problems.tex


\begin{questions}

\question
Consider the following Turing Machine.

\begin{table}[H]
  \centering
  \begin{tabular}{ccccc}
    \toprule
    State	& Input	& Write & Move & Next \\
    \midrule
    \multirow{3}{*}{0} 
    	& B & B &  & Accept \\
      & 0 & 0 & L & 0 \\
      & 1 & 1 & L & 1 \\
    \midrule
    \multirow{3}{*}{1}
    	 & B & B &  & Fail \\
       & 0 & 0 & L & 1 \\
       & 1 & 1 & L & 0 \\
    \bottomrule
    \hline
  \end{tabular}
\end{table}
Determine what happens when the Turing Machine is run with the following inputs initially on the tape.
\begin{parts}
  \part 0001
  \part 0111
  \part 0110
  \part 0101010001
  \part 00000000000000111
  \part 00
  \part 
\end{parts}


\begin{solution}
\begin{parts}
  \part Fail
  \part Fail
  \part Accept
  \part Accept
  \part Fail
  \part Accept
  \part Accept
\end{parts}
\end{solution}


\question
Give the state table for a Turing Machine that appends a parity bit to a tape with a string of consecutive 0's and 1's.

\question
Construct a Turing Machine to compute the sequence $0\_1\_0\_1\_0\_1\ldots$, that is, 0 blank 1 blank 0 blank, etc~\cite{turing}.

\question
Give the state table for a Turing Machine that multiplies a string of consecutive 0's and 1's by 2. The machine should treat the initial contents of the tape as a natural number written in binary form.

\question
Give the state table for a Turing Machine that adds 1 to a string of consecutive 0's and 1's.

\question
Give the state table for a Turing Machine that subtracts 1 to a string of consecutive 0's and 1's.


\end{questions}

