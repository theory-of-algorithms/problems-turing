\documentclass[addpoints,12pt]{exam}

\makeatletter
\expandafter\providecommand\expandafter*\csname ver@framed.sty\endcsname
{2003/07/21 v0.8a Simulated by exam}
\makeatother

\usepackage{xcolor}
\usepackage{minted}
\usepackage[utf8]{inputenc}
\usepackage{tikz}
\usepackage{caption}
\usepackage{gensymb}
\usepackage{lmodern}
\usepackage{multirow}
\usepackage{booktabs}
\usepackage{array}
\usepackage{amsmath}
\usepackage{amsfonts}


\renewcommand{\refname}{\selectfont\normalsize References} 
\pagestyle{headandfoot}

\header{\textbf{Problem Sheet: Turing machines}}{}{Theory of Algorithms}
\footer{}{Page \thepage\ of \numpages}{}
\marksnotpoints
\pointsinrightmargin

\begin{coverpages}
\end{coverpages}

\begin{document}

\begin{questions}

\question
Consider the following Turing Machine.
\begin{table}[H]
  \centering
  \begin{tabular}{ccccc}
    \toprule
        State & Input & Write & Move & Next \\
    \midrule
        $q_0$ & $\sqcup$ & $\sqcup$ & L & $q_a$ \\
        $q_0$ & 0 & 0 & R & $q_0$ \\
        $q_0$ & 1 & 1 & R & $q_1$ \\
    \midrule
        $q_1$ & $\sqcup$ & $\sqcup$ & L & $q_f$ \\
        $q_1$ & 0 & 0 & R & $q_1$ \\
        $q_1$ & 1 & 1 & R & $q_0$ \\
    \bottomrule
    \hline
  \end{tabular}
\end{table}
Determine what happens when the Turing Machine is run with the following inputs initially on the tape.
\begin{parts}
  \part 0001
  \part 0111
  \part 0110
  \part 0101010001
  \part 00000000000000111
  \part 00
  \part 
\end{parts}

\begin{solution}
\begin{parts}
  \part Fail
  \part Fail
  \part Accept
  \part Accept
  \part Fail
  \part Accept
  \part Accept
\end{parts}
\end{solution}


\question
Give the state table for a Turing Machine that appends a parity bit to a tape with a string of consecutive 0's and 1's.

\begin{solution}
\begin{table}[H]
  \centering
  \begin{tabular}{ccccc}
    \toprule
        State & Input & Write & Move & Next \\
    \midrule
        $q_0$ & $\sqcup$ & 0 & L & $q_a$ \\
        $q_0$ & 0 & 0 & R & $q_0$ \\
        $q_0$ & 1 & 1 & R & $q_1$ \\
    \midrule
        $q_1$ & $\sqcup$ & 1 & L & $q_f$ \\
        $q_1$ & 0 & 0 & R & $q_1$ \\
        $q_1$ & 1 & 1 & R & $q_0$ \\
    \bottomrule
    \hline
  \end{tabular}
\end{table}
\end{solution}

\question
Construct a Turing Machine to compute the sequence $0 \sqcup 1 \sqcup 0 \sqcup 1 \sqcup 0 \sqcup \ldots$, that is, 0 blank 1 blank 0 blank, etc~\cite{turing37}.

\question
Give the state table for a Turing Machine that multiplies a string of consecutive 0's and 1's by 2.
The machine should treat the initial contents of the tape as a natural number written in binary form, with the least significant bit at the end.
That is, if the contents of the tape are $01101$, then the right-most 1 represents the number 1, the middle 1 represents the number 4 and the left-most 1 represents the number 8.
Then the number on the tape is $8+4+1=13$.


\question
Give the state table for a Turing Machine that multiplies a string of consecutive 0's and 1's by 2.
The machine should treat the initial contents of the tape as a natural number written in binary form, with the most significant bit at the end.
That is, if the contents of the tape are $01101$, then the right-most 1 represents the number 16, the middle 1 represents the number 4 and the left-most 1 represents the number 2.
Then the number of the tape is $2+4+16=22$.

\question
Give the state table for a Turing Machine that adds 1 to a string of consecutive 0's and 1's.

\question
Give the state table for a Turing Machine that subtracts 1 to a string of consecutive 0's and 1's.


\question
List all words of length at most three in $\Sigma^*$ where $\Sigma$ is:
\begin{parts}
  \part $\{ 0, 1\}$
  \part $\{ a, b, c\}$
  \part $\{ \}$
\end{parts}

\question
Design a Turing machine to recognise the language $\{ 0^n 1^n | n \geq 1 \}$.

\question
Design a Turing machine to recognise the language $\{ ww | w \in \{ 0,1 \}^* \}$

\question
Design a Turing machine to recognise the language $\{ a^ib^jc^k | i,j,k \in \mathbb{N}_0 \}$

\end{questions}


\bibliographystyle{plain}
\bibliography{bibliography}
\end{document}
