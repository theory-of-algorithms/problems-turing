\documentclass[a4paper, 12pt]{exam}

% Exam class is broken without this.
\makeatletter
\expandafter\providecommand\expandafter*\csname ver@framed.sty\endcsname
{2003/07/21 v0.8a Simulated by exam}
\makeatother

% Enables the use of colour.
\usepackage{xcolor}
% Syntax high-lighting for code. Requires Python's pygments.
\usepackage{minted}
% Enables the use of umlauts and other accents.
\usepackage[utf8]{inputenc}
% Diagrams.
\usepackage{tikz}
% Settings for captions, such as sideways captions.
\usepackage{caption}
% Symbols for units, like degrees and ohms.
\usepackage{gensymb}
% Latin modern fonts - better looking than the defaults.
\usepackage{lmodern}
% Allows for columns spanning multiple rows in tables.
\usepackage{multirow}
% Better looking tables, including nicer borders.
\usepackage{booktabs}
% More math symbols.
\usepackage{amssymb}
% More math layouts, equation arrays, etc.
\usepackage{amsmath}
% More math fonts, like mathbb.
\usepackage{amsfonts}
% More theorem environments.
\usepackage{amsthm}
% More column formats for tables.
\usepackage{array}
% Adjust the sizes of box environments.
\usepackage{adjustbox}
% Better looking single quotes in verbatim and minted environments.
\usepackage{upquote}
% URLs.
\usepackage[hidelinks]{hyperref}
% Better blank space decisions.
\usepackage{xspace}
% Better looking tikz trees.
\usepackage{forest}
% For plotting.
\usepackage{pgfplots}

% Various tikz libraries.
\usetikzlibrary{mindmap}
\usetikzlibrary{shadows}
\usetikzlibrary{arrows}
\usetikzlibrary{positioning}
\usetikzlibrary{chains}
\usetikzlibrary{fit}
\usetikzlibrary{shapes}
\usetikzlibrary{decorations.markings}
\usetikzlibrary{calc}
\usetikzlibrary{arrows.meta}

% GMIT colours.
\definecolor{gmitblue}{RGB}{20,134,225}
\definecolor{gmitred}{RGB}{220,20,60}
\definecolor{gmitgrey}{RGB}{67,67,67}

% Rename Bibliography to a smaller "Refereces".
\renewcommand{\refname}{\selectfont\normalsize References} 

% Stop minted high-lighting errors.
\makeatletter
\expandafter\def\csname PYGdefault@tok@err\endcsname{\def\PYGdefault@bc##1{{\strut ##1}}}
\makeatother

% Set the header and footer.
\pagestyle{headandfoot}
\header{\textbf{Problem Sheet: Turing machines}}{}{Theory of Algorithms}
\footer{}{Page \thepage\ of \numpages}{}

% Change some things to do with marks.
\marksnotpoints
\pointsinrightmargin

% Empty cover page.
\begin{coverpages}
\end{coverpages}

% Print answers
\printanswers

\begin{document}

\begin{questions}

\question
  Consider the following Turing Machine.
  \begin{table}[H]
    \centering
    \begin{tabular}{ccccc}
      \toprule
          State & Input & Write & Move & Next \\
      \midrule
          $q_0$ & $\sqcup$ & $\sqcup$ & L & $q_a$ \\
          $q_0$ & 0 & 0 & R & $q_0$ \\
          $q_0$ & 1 & 1 & R & $q_1$ \\
      \midrule
          $q_1$ & $\sqcup$ & $\sqcup$ & L & $q_f$ \\
          $q_1$ & 0 & 0 & R & $q_1$ \\
          $q_1$ & 1 & 1 & R & $q_0$ \\
      \bottomrule
      \hline
    \end{tabular}
  \end{table}
  Determine what happens when the Turing Machine is run with the following inputs initially on the tape.
  \begin{parts}
    \part 0001
    \part 0111
    \part 0110
    \part 0101010001
    \part 00000000000000111
    \part 00
    \part 
  \end{parts}
  \begin{solution}
  \begin{parts}
    \part Fail
    \part Fail
    \part Accept
    \part Accept
    \part Fail
    \part Accept
    \part Accept
  \end{parts}
  \end{solution}


\question
  Give the state table for a Turing Machine that appends a parity bit to a tape with a string of consecutive 0's and 1's.
  \begin{solution}
    \begin{table}[H]
      \centering
      \begin{tabular}{ccccc}
        \toprule
            State & Input & Write & Move & Next \\
        \midrule
            $q_0$ & $\sqcup$ & 0 & L & $q_a$ \\
            $q_0$ & 0 & 0 & R & $q_0$ \\
            $q_0$ & 1 & 1 & R & $q_1$ \\
        \midrule
            $q_1$ & $\sqcup$ & 1 & L & $q_f$ \\
            $q_1$ & 0 & 0 & R & $q_1$ \\
            $q_1$ & 1 & 1 & R & $q_0$ \\
        \bottomrule
        \hline
      \end{tabular}
    \end{table}
  \end{solution}


\question
  Construct a Turing Machine to compute the sequence $0 \sqcup 1 \sqcup 0 \sqcup 1 \sqcup 0 \sqcup \ldots$, that is, 0 blank 1 blank 0 blank, etc~\cite{turing37}.
  Give the state table for a Turing Machine that appends a parity bit to a tape with a string of consecutive 0's and 1's.
  \begin{solution}
    \begin{table}[H]
      \centering
      \begin{tabular}{ccccc}
        \toprule
            State & Input & Write & Move & Next \\
        \midrule
            $q_0$ & $\sqcup$ & 0 & R & $q_1$ \\
            $q_0$ & 0 & 0 & R & $q_f$ \\
            $q_0$ & 1 & 1 & R & $q_f$ \\
        \midrule
            $q_1$ & $\sqcup$ & $\sqcup$ & R & $q_2$ \\
            $q_1$ & 0 & 0 & R & $q_f$ \\
            $q_1$ & 1 & 1 & R & $q_f$ \\
        \midrule
            $q_2$ & $\sqcup$ & 1 & R & $q_3$ \\
            $q_2$ & 0 & 0 & R & $q_f$ \\
            $q_2$ & 1 & 1 & R & $q_f$ \\
        \midrule
            $q_3$ & $\sqcup$ & $\sqcup$ & R & $q_0$ \\
            $q_3$ & 0 & 0 & R & $q_f$ \\
            $q_3$ & 1 & 1 & R & $q_f$ \\
        \bottomrule
        \hline
      \end{tabular}
    \end{table}
  \end{solution}


\question
  Give the state table for a Turing Machine that multiplies a string of consecutive 0's and 1's by 2.
  The machine should treat the initial contents of the tape as a natural number written in binary form, with the least significant bit at the end.
  That is, if the contents of the tape are $01101$, then the right-most 1 represents the number 1, the middle 1 represents the number 4 and the left-most 1 represents the number 8.
  Then the number on the tape is $8+4+1=13$.
  \begin{solution}
    \begin{table}[H]
      \centering
      \begin{tabular}{ccccc}
        \toprule
            State & Input & Write & Move & Next \\
        \midrule
            $q_0$ & $\sqcup$ & 0 & R & $q_a$ \\
            $q_0$ & 0 & 0 & R & $q_0$ \\
            $q_0$ & 1 & 1 & R & $q_0$ \\
        \bottomrule
        \hline
      \end{tabular}
    \end{table}
  \end{solution}


\question
  Give the state table for a Turing Machine that multiplies a string of consecutive 0's and 1's by 2.
  The machine should treat the initial contents of the tape as a natural number written in binary form, with the most significant bit at the end.
  That is, if the contents of the tape are $01101$, then the right-most 1 represents the number 16, the middle 1 represents the number 4 and the left-most 1 represents the number 2.
  Then the number of the tape is $2+4+16=22$.
  \begin{solution}
    \begin{table}[H]
      \centering
      \begin{tabular}{ccccc}
        \toprule
            State & Input & Write & Move & Next \\
        \midrule
            $q_0$ & $\sqcup$ & $\sqcup$ & L & $q_1$ \\
            $q_0$ & 0 & 0 & R & $q_0$ \\
            $q_0$ & 1 & 1 & R & $q_0$ \\
        \midrule
            $q_1$ & $\sqcup$ & $\sqcup$ & R & $q_4$ \\
            $q_1$ & 0 & $\sqcup$ & R & $q_2$ \\
            $q_1$ & 1 & $\sqcup$ & R & $q_3$ \\
        \midrule
            $q_2$ & $\sqcup$ & 0 & L & $q_0$ \\
            $q_2$ & 0 & 0 & R & $q_f$ \\
            $q_2$ & 1 & 1 & R & $q_f$ \\
        \midrule
            $q_3$ & $\sqcup$ & 1 & L & $q_0$ \\
            $q_3$ & 0 & 0 & R & $q_f$ \\
            $q_3$ & 1 & 1 & R & $q_f$ \\
        \midrule
            $q_4$ & $\sqcup$ & 0 & R & $q_a$ \\
            $q_4$ & 0 & 0 & R & $q_f$ \\
            $q_4$ & 1 & 1 & R & $q_f$ \\
        \bottomrule
        \hline
      \end{tabular}
    \end{table}
  \end{solution}


\question
  Give the state table for a Turing Machine that adds 1 to a string of consecutive 0's and 1's, where the least significant digit is on the right of the input.
  \begin{solution}
    \begin{table}[H]
      \centering
      \begin{tabular}{ccccc}
        \toprule
            State & Input & Write & Move & Next \\
        \midrule
            $q_0$ & $\sqcup$ & $\sqcup$ & L & $q_1$ \\
            $q_0$ & 0        & 0        & R & $q_0$ \\
            $q_0$ & 1        & 1        & R & $q_0$ \\
        \midrule
            $q_1$ & $\sqcup$ & 1        & L & $q_a$ \\
            $q_1$ & 0        & 1        & L & $q_a$ \\
            $q_1$ & 1        & 0        & L & $q_1$ \\
        \bottomrule
        \hline
      \end{tabular}
    \end{table}
  \end{solution}


\question
  Give the state table for a Turing Machine that subtracts 1 to a string of consecutive 0's and 1's, where the least significant digit is on the right of the input.
  \begin{solution}
    \begin{table}[H]
      \centering
      \begin{tabular}{ccccc}
        \toprule
            State & Input & Write & Move & Next \\
        \midrule
            $q_0$ & $\sqcup$ & $\sqcup$ & L & $q_1$ \\
            $q_0$ & 0        & 0        & R & $q_0$ \\
            $q_0$ & 1        & 1        & R & $q_0$ \\
        \midrule
            $q_1$ & $\sqcup$ & $\sqcup$ & L & $q_a$ \\
            $q_1$ & 0        & 1        & L & $q_1$ \\
            $q_1$ & 1        & 0        & L & $q_a$ \\
        \bottomrule
        \hline
      \end{tabular}
    \end{table}
  \end{solution}


\question
  List all words of length at most three in $\Sigma^*$ where $\Sigma$ is:
  \begin{parts}
    \part $\{ 0, 1 \}$
    \part $\{ a, b, c \}$
    \part $\{ \}$
  \end{parts}
  \begin{solution}
    \begin{parts}
      \part $\{ \epsilon, 0, 1, 00, 01, 10, 11, 000, 001, 010, 011, 100, 101, 110, 111 \}$
      \part $\{ \epsilon, a, b, c, aa, ab, ac, ba, bb, bc, ca, cb, cc, abc, acb, bac, bca, cab, cba \}$
      \part $\{ \epsilon \}$
    \end{parts}
  \end{solution}

\question
  Design a Turing machine to recognise the language $\{ 0^n 1^n \mid n \geq 0 \}$.
  \begin{solution}
    \begin{table}[H]
      \centering
      \begin{tabular}{ccccc}
        \toprule
            State & Input & Write & Move & Next \\
        \midrule
            $q_0$ & $\sqcup$ & $\sqcup$ & R & $q_a$ \\
            $q_0$ & 0        & $\sqcup$ & R & $q_1$ \\
            $q_0$ & 1        & 1        & R & $q_f$ \\
        \midrule
            $q_1$ & $\sqcup$ & $\sqcup$ & L & $q_2$ \\
            $q_1$ & 0        & 0        & R & $q_1$ \\
            $q_1$ & 1        & 1        & R & $q_1$ \\
        \midrule
            $q_2$ & $\sqcup$ & $\sqcup$ & L & $q_f$ \\
            $q_2$ & 0        & 0        & R & $q_f$ \\
            $q_2$ & 1        & $\sqcup$ & L & $q_3$ \\
        \midrule
            $q_3$ & $\sqcup$ & $\sqcup$ & R & $q_0$ \\
            $q_3$ & 0        & 0        & L & $q_3$ \\
            $q_3$ & 1        & 1        & L & $q_3$ \\
        \bottomrule
        \hline
      \end{tabular}
    \end{table}
  \end{solution}


\question
  Design a Turing machine to recognise the language $\{ ww^R \mid w \in \{ 0,1 \}^* \}$ where $w^R$ is $w$ reversed.
  For example, when $w = 101011$ then $w^R = 110101$.
  \begin{solution}
    \begin{table}[H]
      \centering
      \begin{tabular}{ccccc}
        \toprule
            State & Input & Write & Move & Next \\
        \midrule
            $q_0$ & $\sqcup$ & $\sqcup$ & R & $q_a$ \\
            $q_0$ & 0        & $\sqcup$ & R & $q_1$ \\
            $q_0$ & 1        & $\sqcup$ & R & $q_3$ \\
        \midrule
            $q_1$ & $\sqcup$ & $\sqcup$ & L & $q_2$ \\
            $q_1$ & 0        & 0        & R & $q_1$ \\
            $q_1$ & 1        & 1        & R & $q_1$ \\
        \midrule
            $q_2$ & $\sqcup$ & $\sqcup$ & L & $q_f$ \\
            $q_2$ & 0        & $\sqcup$ & L & $q_5$ \\
            $q_2$ & 1        & 1        & L & $q_f$ \\
        \midrule
            $q_3$ & $\sqcup$ & $\sqcup$ & L & $q_4$ \\
            $q_3$ & 0        & 0        & R & $q_3$ \\
            $q_3$ & 1        & 1        & R & $q_3$ \\
        \midrule
            $q_4$ & $\sqcup$ & $\sqcup$ & L & $q_f$ \\
            $q_4$ & 0        & 0        & L & $q_f$ \\
            $q_4$ & 1        & $\sqcup$ & L & $q_5$ \\
        \midrule
            $q_5$ & $\sqcup$ & $\sqcup$ & R & $q_0$ \\
            $q_5$ & 0        & 0        & L & $q_5$ \\
            $q_5$ & 1        & 1        & L & $q_5$ \\
        \bottomrule
        \hline
      \end{tabular}
    \end{table}
  \end{solution}


\question
  Design a Turing machine to recognise the language $\{ a^ib^jc^k \mid i,j,k \in \mathbb{N}_0 \}$
  \begin{solution}
    \begin{table}[H]
      \centering
      \begin{tabular}{ccccc}
        \toprule
            State & Input & Write & Move & Next \\
        \midrule
            $q_0$ & $\sqcup$ & $\sqcup$ & R & $q_a$ \\
            $q_0$ & $a$      & $a$      & R & $q_0$ \\
            $q_0$ & $b$      & $b$      & R & $q_1$ \\
            $q_0$ & $c$      & $c$      & R & $q_2$ \\
        \midrule
            $q_0$ & $\sqcup$ & $\sqcup$ & R & $q_a$ \\
            $q_0$ & $a$      & $a$      & R & $q_f$ \\
            $q_0$ & $b$      & $b$      & R & $q_1$ \\
            $q_0$ & $c$      & $c$      & R & $q_2$ \\
        \midrule
            $q_0$ & $\sqcup$ & $\sqcup$ & R & $q_a$ \\
            $q_0$ & $a$      & $a$      & R & $q_f$ \\
            $q_0$ & $b$      & $b$      & R & $q_f$ \\
            $q_0$ & $c$      & $c$      & R & $q_2$ \\
        \bottomrule
        \hline
      \end{tabular}
    \end{table}
  \end{solution}

\end{questions}
\bibliographystyle{plain}
\bibliography{bibliography}
\end{document}
